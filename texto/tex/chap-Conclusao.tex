%!TEX root = ../main_text.tex
%\chapter{Experimentos e Resultados} \label{chap:experimentos}

\chapter{Conclusões Parciais e Trabalho Futuro} \label{chap:conclusao}

   Neste capítulo serão descritas as conclusões parciais da pesquisa realizada até o momento e as próximas etapas a serem executadas.   

   \section{Conclusões}
      %demora da pesquisa em wearables
      Mesmo a tecnologia embarcada tenha tido um grande avanço nos últimos tempos sendo utilizada nos mais diversos fins, sistemas \wearables\ tiveram seu primeiro aparecimento em pesquisas científicas em \citeyear{Mann1996} por \citeauthor{Mann1996}.
      Isso, por causa de empecilhos tecnológicos e ferramentais como a necessidade da miniaturização, mobilidade e eficiência energética da tecnologia dos equipamentos utilizados na qual, sem tais especificações, seria inviável a utilização do \wearable.
      
      %não tem particionamento para wearables
      Existe várias pesquisas relacionadas à área de sistemas embarcados no âmbito de particionamento de \hs, inclusive utilizando FPGAs como meio.
      Entretanto, até o momento deste, não foi encontrado nenhum relato de experiências científicas sobre o particionamento para sistemas \wearables\ com utilização de recursos de \hardwares\ reconfiguráveis, tema tratado neste documento.
      
      % objetivos concluídos até então
      Objetivos realizados até o presente momento:
      
      \begin{itemize}
         \item Introdução de sistemas computacionais \wearables\ e apresentação do problema de particionamento \hs\ no âmbito de de sistemas computacionais embutidos, com foco em sistemas \wearables;
         
         \item Apresentação de:
         \begin{itemize}
            \item Principais soluções apresentadas ao logo dos anos e as utilizadas atualmente;
            \item Ferramentas HLS como LegUp e OpenCL para a geração de sistemas computacionais que usufruem de aceleradores em \hardware.
         \end{itemize}
         
         \item Apresentação da abordagem metodológica a ser utilizada para a procura da solução do problema de particionamento \hs\ apresentado.
      \end{itemize}
      
      
   \section{Trabalho Futuro}
      A seguir será apresentado os tópicos relativos às próximas etapas a serem realizadas neste trabalho.
      
      \begin{itemize}
         \item Geração de HDL, utilizando as ferramentas automatizadas quando aplicável:
         \begin{itemize}
            \item LegUp;
            \item OpenCL.
         \end{itemize}
         
      \item Realizar-se-á análises de algoritmos situados em sistemas diferentes, sendo esses:
         \begin{itemize}
            \item \textbf{Totalmente em nível de \software:} sem auxílio de qualquer acelerador como ao utilizar a plataforma Beagle Bone como sendo o sistema embarcado completo para execução em aplicações \wearables; e em
            
            \item \textbf{Híbridos:} formados de processadores e aceleradores utilizando dos recursos de dispositivos reconfiguráveis  a fim de prover maior \speedup\ utilizando recursos de \hardware.
         \end{itemize}
         
         \item A análise de desempenho será feita com os dados obtidos dos dois sistemas citados acima.
         Executando o projeto nos dois sistemas, em nível de \software\ e híbrido, é possível relacionar o ganho de desempenho realizando as operações apresentadas na Seção~\ref{sec:ganho_performance}.
                  
         
      	\item Redução do tempo de seu desenvolvimento até a disposição do produto ao mercado utilizando os recursos que o desenvolvimento com \hardware\ reconfigurável proporciona. %Wolf1994.
         \end{itemize}
      